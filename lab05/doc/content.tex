% Содержимое отчета по курсу Анализ алгоритмов

\aaunnumberedsection{ВВЕДЕНИЕ}{sec:intro}

Цель работы --- получение навыка организации параллельных вычислений по конвейерному принципу. 

Задачи работы: 
\begin{itemize}
    \item анализ возможностей организации параллельных вычислений по конвейерному принципу;
    \item разработка алгоритма, осуществляющего параллельную выгрузку страниц интернет ресурса по конвейерному принципу;
    \item создание ПО, реализующего разработанный алгоритм;
    \item исследование характеристик созданного ПО.
\end{itemize}

\aasection{Входные и выходные данные}{sec:input-output}

Входными данными являются ссылка на интернет ресурс, содержащий кулинарные рецепты, и количество рецептов, которые необходимо сохранить. Выходными данными являются база данных, содержащая необходимое количество сохраненных рецептов, информация о среднем времени, проведенном задачей на каждом из этапов конвейера и в каждой очереди, и лог выполненных над каждой задачей действий.

\aasection{Преобразование входных данных в выходные}{sec:algorithm}

Программа считывает входные данные и выполняет поиск необходимого количества ссылок на страницы интернет ресурса, содержащих кулинарные рецепты, при помощи библиотеки curl~\cite{curl}, а затем генерирует задачи для каждой из полученных ссылок. Для каждой задачи выполняется загрузка файла страницы, извлечение из него необходимой информации при помощи регулярных выражений~\cite{regex} и запись в базу данных при помощи библиотеки SQLite~\cite{sqlite}, а затем логирование времени выполнения каждого из этапов и закрытие задачи. Каждый из перечисленных этапов (генерация, чтение, извлечение данных, запись и логирование с завершением) выполняется в отдельном потоке по конвейерному принципу, поэтому между выполнением этих этапов задача помещается в соответствующие очереди. Временные метки начала и завершения выполнения каждого из этапов каждой задачей документируется в логе.

\aasection{Примеры работы программы}{sec:demo}

На рисунке~\ref{fig:example} представлен пример работы программы.

\begin{figure}[H]
    \centering
    \includegraphics[width=0.85\linewidth]{report//inc//img/example.png}
    \caption{Пример работы программы}
    \label{fig:example}
\end{figure}

\aasection{Тестирование}{sec:tests}

В таблице~\ref{tbl:tests} представлены функциональные тесты для разработанного ПО. Все тесты пройдены успешно.


\begin{longtable}{|p{.2\textwidth - 2\tabcolsep}|p{.24\textwidth - 2\tabcolsep}|p{.28\textwidth - 2\tabcolsep}|p{.28\textwidth - 2\tabcolsep}|}
  \caption{Функциональные тесты}\label{tbl:tests} \\\hline
  № теста & Входные данные & Полученные выходные данные & Ожидаемые выходные данные                                          \\\hline
  \endfirsthead
  \caption{Функциональные тесты (продолжение)} \\\hline
  № теста & Входные данные & Полученные выходные данные  & Ожидаемые выходные данные                                                 \\\hline
  \endhead
  \endfoot
  1                                           & 1 & База данных recipes с 1 записью & База данных recipes с 1 записью \\\hline
  2                                           & 0 & Пустая база данных & Пустая база данных \\\hline
  3                                           & 10 & База данных recipes с 10 записями & База данных recipes с 10 записями \\\hline
\end{longtable}

\aasection{Описание исследования}{sec:study}

В ходе исследования необходимо получить лог выполненных действий и замерить среднее время обработки задачи на каждой из стадий, среднее время ожидания в каждой из очередей и среднее время жизни задачи.

В таблице~\ref{tbl:log} приведен лог работы программы для трех задач, где:

\begin{itemize}
    \item в первую очередь процесс попадает по окончании процесса генерации и находится в ней в ожидании начала процесса чтения;
    \item во вторую очередь процесс попадает по окончании процесса чтения и находится в ней в ожидании начала процесса извлечения данных;
    \item в третью очередь процесс попадает по окончании процесса извлечения данных и находится в ней в ожидании начала процесса записи в базу данных;
    \item в четвертую очередь процесс попадает по окончании процесса записи в базу данных и находится в ней в ожидании начала процесса логирования и закрытия задачи.
\end{itemize}

\begin{longtable}{|p{.33\textwidth - 2\tabcolsep}|p{.22\textwidth - 2\tabcolsep}|p{.44\textwidth - 2\tabcolsep}|}
	\caption{Лог работы программы (начало)}\label{tbl:log}
	\\
	\hline
	Метка времени, мкс & ID задачи  & Событие \\
	\hline
	\endfirsthead
	\caption{Лог работы программы (продолжение)}
	\\
	\hline
	Метка времени, мкс   & ID задачи  & Событие     \\
	\hline
	\endhead
	\hline
	\endfoot
	\endlastfoot
	\hline
	26873813647100  &1& started generation process\\ \hline
26873813658400  &1& added in first queue\\ \hline
26873813661200  &1& started reading process\\ \hline
26873813672800  &2& started generation process\\ \hline
26873813685900  &2& added in first queue\\ \hline
26873813711800  &3& started generation process\\ \hline
26873813715200  &3& added in first queue\\ \hline
26873946985400  &1& added in second queue\\ \hline
26873946988900  &2& started reading process\\ \hline
26873946989000  &1& started extraction process\\ \hline
26874106097500  &2& added in second queue\\ \hline
26874106099400  &3& started reading process\\ \hline
26874233916000  &3& added in second queue\\ \hline
26875073214900  &1& added in third queue\\ \hline
26875073216100  &2& started extraction process\\ \hline
26875073216900  &1& started writing process\\ \hline
26875101161000  &1& added in forth queue\\ \hline
26875101164000  &1& started destruction process\\ \hline
26875101166200  &1& ended destruction process\\ \hline
26876177932300  &2& added in third queue\\ \hline
26876177934100  &3& started extraction process\\ \hline
26876177935700  &2& started writing process\\ \hline
26876235905400  &2& added in forth queue\\ \hline
26876235908900  &2& started destruction process\\ \hline
26876235911100  &2& ended destruction process\\ \hline
26877299647100  &3& added in third queue\\ \hline
26877299650000  &3& started writing process\\ \hline
26877359864200  &3& added in forth queue\\ \hline
26877359869600  &3& started destruction process\\ \hline
26877359872000  &3& ended destruction process\\ \hline

\end{longtable}

Среднее время обработки задачи на каждой из стадий, среднее время ожидания в каждой из очередей, среднее время жизни задачи при выполнении 50 задач указаны в таблице~\ref{tbl:avg_time}.

\begin{longtable}{|p{.77\textwidth - 2\tabcolsep}|p{.23\textwidth - 2\tabcolsep}|}
    \caption{Среднее время прохождения этапов конвейера}\label{tbl:avg_time}
	\\
	\hline
	Этап обработки задачи & Среднее время обработки, с \\
	\hline
	\endfirsthead
	\caption{Среднее время прохождения этапов конвейера (продолжение)}
	\\
	\hline
	Этап обработки задачи & Среднее время обработки, с     \\
	\hline
	\endhead
	\hline
	\endfoot
	\endlastfoot
	\hline
	Среднее время существования задачи & 3.298$\cdot 10$\\ \hline
	Среднее время генерации задачи & 4.712$\cdot 10^{-6}$ \\ \hline
	Среднее время чтения данных & 1.727$\cdot 10^{-1}$ \\ \hline
	Среднее время извлечения данных & 1.278 \\ \hline
	Среднее время записи данных & 8.303$\cdot 10^{-2}$ \\ \hline
	Среднее время логирования и завершения задачи & 2.256$\cdot 10^{-6}$ \\ \hline
	Среднее время ожидания в первой очереди & 4.126 \\ \hline
	Среднее время ожидания во второй очереди & 2.732$\cdot 10^{1}$ \\ \hline
	Среднее время ожидания в третьей очереди & 1.161$\cdot 10^{-4}$ \\ \hline
	Среднее время ожидания в четвертой очереди & 8.838$\cdot 10^{-6}$ \\ \hline
	
\end{longtable}

В результате исследования сделан вывод, что выполнение различных этапов обработки разных задач происходит параллельно, но этапы выполнения одной задачи --- последовательно. При этом время ожидания в очереди выполнения очередного этапа пропорционально времени выполнения этого этапа.

\aaunnumberedsection{ЗАКЛЮЧЕНИЕ}{sec:outro}

Цель работы достигнута. Решены все поставленные задачи: 
\begin{itemize}
    \item анализ возможностей организации параллельных вычислений по конвейерному принципу;
    \item разработка алгоритма, осуществляющего параллельную выгрузку страниц интернет ресурса по конвейерному принципу;
    \item создание ПО, реализующего разработанный алгоритм;
    \item исследование характеристик созданного ПО.
\end{itemize}