\chapter*{ЗАКЛЮЧЕНИЕ}
\addcontentsline{toc}{chapter}{ЗАКЛЮЧЕНИЕ}

В результате исследования сделан вывод, что на малых размерах графов для решении задачи коммивояжера допустимо использования алгоритма полного перебора, но на графах с большим количеством вершин муравьиный алгоритм работает значительно быстрее. Но для корректной работы муравьиного алгоритма необходимо подобрать правильные параметры, в противном случае найденный маршрут может оказаться не оптимальным.

Цель работы достигнута. В ходе выполнения данной лабораторной работы были решены следующие задачи:

\begin{enumerate}
    \item реализация алгоритмов решения задачи коммивояжера --- полного перебора и муравьиного алгоритма;
    \item исследование ресурсной эффективности реализованных алгоритмов для разных параметров;
    \item описание и обоснование полученных результатов в отчете о выполненной лабораторной работе, выполненного как расчётно--пояснительная записка к работе.
\end{enumerate}
