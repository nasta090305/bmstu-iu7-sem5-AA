\chapter{Исследовательский раздел}\label{ch:research}

\section{Технические характеристики}

Технические характеристики устройства, на котором выполнялись замеры по времени, представлены далее.

\begin{itemize}
	\item Процессор: Intel(R) Core(TM) i5-8250U CPU @ 1.60ГГц   1.80 ГГц.
	\item Оперативная память: 8 ГБайт.
	\item Операционная система: Windows 11 Home версии 23H2.
\end{itemize}

При замерах времени ноутбук был включен в сеть электропитания и был нагружен только системными приложениями.

\section{Временные характеристики}

Для замеров времени были использованы функции библиотеки time~\cite{pytime}.

Исследование произведено на графах с количеством вершин 1 до 10. Для каждого размера графа замеры проводились 50 раз. Средние значения полученных результатов представлены в таблице~\ref{table:time}.

\begin{center}
  \captionsetup{justification=raggedright,singlelinecheck=off}
  \begin{longtable}[c]{|c|c|c|}
    \caption{Среднее время выполнения алгоритмов, мкс\label{table:time}} \\ \hline
    Кол-во вершин графа & Полный перебор & Муравьиный алгоритм\\
    \hline
    1 & 1 & 8 \\ \hline
    2 & 2 & 18 \\ \hline
    3 & 7 & 39 \\ \hline
    4 & 23 & 68 \\ \hline
    5 & 59 & 93 \\ \hline
    6 & 143 & 138\\ \hline
    7 & 260 & 223\\ \hline
    8 & 2959 & 436\\ \hline
    9 & 14874 & 812\\ \hline
    10 & 311094 & 1448 \\ \hline
  \end{longtable}
\end{center}

Таким образом, был сделан вывод, что на небольших значениях количества вершин (до 5 включительно) алгоритм полного перебора работает быстрее муравьиного, но на более больших количествах вершин скорость муравьиного алгоритма начинает значительно превышать скорость алгоритма полного перебора.

\section{Параметризация муравьиного алгоритма}

В ходе исследования муравьиного алгоритма изменялся входной параметр  $a$ --- коэффициент жадности. Исследование проводилось на трех графах, каждый из которых состоял из 10 вершин. Для каждого графа и каждого набора параметров алгоритм запускался 20 раз, были вычислены максимальные, средние и медианные значения отклонения длины найденного маршрута от эталонной длины. Результаты исследования представлены в Приложении А. Из полученных данных следует, что с увеличением коэффициента жадности наблюдается снижение максимальных, минимальных и средних отклонений. Это указывает на то, что точность муравьиного алгоритма возрастает при увеличении коэффициента жадности.

Таким образом рекомендованные значения параметров:

\begin{itemize}
    \item коэффициент жадности = 0.9;
    \item коэффициент стадности = 0.1;
    \item количество итераций = 500.
\end{itemize}

\clearpage

