\chapter{Аналитический раздел}\label{ch:analysis}

Задача коммивояжера --- это классическая задача в теории графов и комбинаторной оптимизации~\cite{task}. Она формулируется следующим образом:

Предположим, что у вас есть набор городов и известные расстояния между каждой парой городов. Задача состоит в том, чтобы найти самый короткий маршрут, который начинается в одном городе, проходит через все остальные города ровно один раз и возвращается в исходный город.

\section{Алгоритм полного перебора}

Для решения задачи коммивояжера для n городов методом полного перебора, необходимо построить все $n!$ возможных путей, посчитать для каждого длину и, сравнив полученные значения, выбрать один из путей минимальной длины. 

Таким образом, данный алгоритм обладает высокой сложностью $O(n!)$, что делает его неэффективным, а при больших значениях n --- технически невозможным.

\section{Муравьиный алгоритм}

Муравьиный алгоритм --- это метаэвристический метод, вдохновленный поведением муравьев, который используется для решения различных задач оптимизации, включая задачу коммивояжера. Основная идея заключается в имитации процесса поиска пищи муравьями и их способности оставлять феромоны, которые влияют на поведение других муравьев. Реализация же заключается в моделировании нескольких, заданных параметром $t_{max}$, дней жизни муравьиной колонии.

Каждый день муравьи выходят из вершин графа (количество муравьев равно количеству вершин в графе) и находят по одному решению каждый. Взаимодействуют они при помощи феромона, который оставляют на пройденных маршрутах. Количество феромонов зависит от качества найденного решения (например, длины маршрута). Чем короче маршрут, тем больше феромонов оставляет муравей. Феромоны испаряются со временем согласно формуле~\eqref{equ:evap}, что позволяет избежать прежнего доминирования неэффективных решений.

\begin{equation}\label{equ:evap}
	\tau_{ij}(t+1) = \tau_{ij}(t)\cdot(1-p),
\end{equation}
где $p \in (0;1)$ --- коэффициент испарения.

Матрица феромонов инициализируется начальным значением феромона $\tau_{0}$.

При выборе следующего города муравей учитывает количество феромонов на ребрах и расстояние до следующего города. Вероятность перехода от города i к городу j определяется формулой~\eqref{equ:p}
\begin{equation}\label{equ:p}
        p_{ij} = \begin{cases}
		\frac{\eta_{ij}^{\alpha}\cdot\tau_{ij}^{\beta}}{\sum_{q\notin J_k} \eta^\alpha_{iq}\cdot\tau^\beta_{iq}}, j \notin J_k \\
		0, j \in J_k
	\end{cases}
\end{equation}
где $a$ --- коэффициент жадности, $b$ --- коэффициент стадности, $\tau_{ij}$ --- количество феромонов на ребре $ij$, $\eta_{ij}$ --- привлекательность ребра $ij$, $J_k$ --- список посещенных за текущий день городов.

В конце дня в каждую дугу $i-j$ для всех полученных маршрутов $P$ добавляется $\Delta\tau_{ij}$ феромона, где 
\begin{equation}\label{equ:update_pher}
	\Delta \tau_{ij}(t) = \frac{Q}{\sigma(P)},
\end{equation}
где $Q$ --- количество феромона, которое распределяется на все дуги за день, а $\sigma(P)$ --- сумма всех дуг маршрута $P$.

Этапы работы алгоритма в течение одного дня.

\begin{enumerate}
    \item Каждый муравей оказывается в вершине с номером $k \in [1, n]$ и принимает за $V = \{k\}$ список посещенных в этот день вершин.
    \item Каждый муравей оценивает вероятность перехода в следующий город по формуле~\eqref{equ:p}.
    \item Каждый муравей выбирает город для перехода (город с наибольшей вероятностью перехода).
    \item Если есть непосещенные города, переходим к шагу 2.
    \item Муравей оказался в вершине, с которой начинал путь в начале дня. Происходит испарение феромона по формуле~\eqref{equ:evap}.
    \item Все маршруты $P$ просматриваются и для каждого в составляющие его дуги добавляются феромоны по формуле~\eqref{equ:update_pher}.
    \item Если есть "элитные" муравьи, шаг 6 повторяется для наилучших маршрутов.
\end{enumerate}

Сложность такого алгоритма составляет $O(t_{max} \cdot n^2)$.