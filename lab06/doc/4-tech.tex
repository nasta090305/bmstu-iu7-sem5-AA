\chapter{Технологический раздел}\label{ch:tech}

\section{Средства реализации}

Для реализации данной лабораторной был выбран язык Python~\cite{python}, так как он содержит все необходимые для реализаций алгоритмов инструменты.

\section{Реализация алгоритмов}

Реализации алгоритмов представлены в листингах~\ref{lst:brute_force.py}--\ref{lst:update_phers.py}.

\includelisting
{brute_force.py}
{Реализация алгоритма полного перебора}

\includelisting
{ant_main.py}
{Основная функция муравьиного алгоритма}

\includelisting
{get_path.py}
{Генерация пути одного муравья}

\includelisting
{next_node.py}
{Выбор следующей вершины, в которую будет совершен переход}

\includelisting
{update_phers.py}
{Обновление матрицы феромонов}

\clearpage

\section{Функциональные тесты}

Функциональные тесты приведены в таблице~\ref{table:tests} и были пройдены успешно всеми алгоритмами.

\begin{center}
\captionsetup{justification=raggedright,singlelinecheck=off}
	\begin{longtable}[c]{|c|c|c|c|c|}
		\caption{Функциональные тесты}\label{table:tests} \\ \hline
		Матрица смежности & Ожидаемый результат & Результат программы \\
		\hline
		$ \begin{pmatrix}
			0 &  1 &  4 &  16  \\
			1 &  0 &  9 &  25  \\
			4 &  9 &  0 & 36  \\
			16 &  25 & 36 &  0  \\
		\end{pmatrix}$ &
		54, [0, 2, 1, 3, 0] &
		54, [0, 2, 1, 3, 0] \\ \hline
		
		$ \begin{pmatrix}
			0 & 1 & 2 \\
			3 & 0 & 4 \\
			5 & 6 & 0	
		\end{pmatrix}$ &
		10, [0, 1, 2, 0] &
		10, [0, 1, 2, 0] \\ \hline
		
		$ \begin{pmatrix}
			0 & 1 & 2 & 12 & 3  \\
                3 & 0 & 4 & 43 & 5  \\
                5 & 6 & 0 & 24 & 1  \\
                63 & 43 & 24 & 0 & 2  \\
                3 & 6 & 14 & 4 & 0 
		\end{pmatrix}$ &
		29, [0, 3, 4, 1, 2, 0] &
		29, [0, 3, 4, 1, 2, 0] \\
		\hline
	\end{longtable}
\end{center}
