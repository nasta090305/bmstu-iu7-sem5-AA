\chapter{Конструкторский раздел}\label{ch:design}

\section{Требования к программе}

К программе предъявлены следующие требования:

\begin{itemize}
        \item на вход подается матрица расстояний, описывающая граф;
        \item результатом работы программы является целое число --- длина оптимального пути и массив индексов вершин, описывающий этот путь;
	\item необходимо наличие пользовательского интерфейса для выбора действий и алгоритма работы.
\end{itemize}

\section{Проектирование алгоритмов}

В процессе проектирования программы были разработаны алгоритмы решения задачи коммивояжера, представленные на рисунках~\ref{img:brute_force} и~\ref{img:ants}.

\clearpage

\includeimage
    {brute_force} % Имя файла без расширения (файл должен быть расположен в директории inc/img/)
    {f} % Обтекание (без обтекания)
    {H} % Положение рисунка (см. figure из пакета float)
    {0.7\textwidth} % Ширина рисунка
    {Алгоритм полного перебора} % Подпись рисунка

\clearpage

\includeimage
    {ants}
    {f}
    {H}
    {1\textwidth}
    {Муравьиный алгоритм}

\clearpage
