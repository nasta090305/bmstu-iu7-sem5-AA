\aaunnumberedsection{ВВЕДЕНИЕ}{sec:intro}
Цель данный лабораторной работы заключается в исследовании графовых моделей.

Задачи работы: 
\begin{itemize}
	\item построение графа управления;
	\item построение информационного графа;
	\item построение операционной истории;
	\item построение информационной истории.
\end{itemize}

\aasection{Построение графовых моделей}{}
Фрагмент кода, на основе которого построены графовые модели, приведён в листинге~\ref{lst:func.c}.

\includelistingpretty
    {func.c} % Имя файла с расширением (файл должен быть расположен в директории inc/lst/)
    {c}
    {Фрагмент кода} % Подпись листинга

На основе вышеприведённого кода были построены 4 графовых модели.

На рисунке~\ref{fig:G_C} приведены граф управления и информационный граф.

На рисунке~\ref{fig:O_H} приведена операционная история.

На рисунке~\ref{fig:I_H} приведена информационная история.

\begin{figure}[H]
    \centering
    \includegraphics[width=1\linewidth]{inc//img/CG.png}
    \caption{Граф управления и информационный граф}
    \label{fig:G_C}
\end{figure}

\begin{figure}[H]
    \centering
    \includegraphics[width=1\linewidth]{inc//img/O_I.png}
    \caption{Операционная история}
    \label{fig:O_H}
\end{figure}

\begin{figure}[H]
    \centering
    \includegraphics[width=1\linewidth]{inc/img/I_H.png}
    \caption{Информационная история}
    \label{fig:I_H}
\end{figure}


\newpage
\aaunnumberedsection{ЗАКЛЮЧЕНИЕ}{sec:outro}

Цель работы достигнута. Все поставленные задачи решены: 
\begin{itemize}
	\item построение графа управления;
	\item построение информационного графа;
	\item построение операционной истории;
	\item построение информационной истории.
\end{itemize}
