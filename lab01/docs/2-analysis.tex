\chapter{Аналитический раздел}\label{ch:analysis}

\section{Определение расстояния Левенштейна}

\textbf{Расстояние Левенштейна} между двумя строками --- это минимальное количество операций вставки, удаления и замены, необходимых для превращения одной строки в другую.

Расстояние Левенштейна применяется в теории информации и компьютерной лингвистике для:
\begin{itemize}
	\item исправления ошибок в слове
	\item сравнения текстовых файлов утилитой diff
	\item в биоинформатике для сравнения генов, хромосом и белков
\end{itemize}

\textbf{Расстояние Дамерау--Левенштейна} отличается от расстояния Левенштейна добавлением операции перестановки двух рядом стоящих букв в слове.

\section{Вычисление расстояния Левенштейна}
Пусть $S_N$ и $S_M$ --- две строки длиной N и M cоответственно. Цены всех операций примем равными 1. 
Тогда для этих строк расстояние Левенштейна $d(S_N, S_M) = D(N, M)$, где 
\begin{equation}\label{ed:D}
D(i,j) = \left\{ \begin{array}{ll}
 0, & \textrm{$i = 0, j = 0$}\\
 i, & \textrm{$j = 0, i > 0$}\\
 j, & \textrm{$i = 0, j > 0$}\\
min\{\\
\qquad D(i,j-1)+1,\\
\qquad D(i-1, j) +1, &\textrm{$j>0, i>0$}\\
\qquad D(i-1, j-1) + m(S_N[i], S_M[j])\\
\},
  \end{array} \right.
\end{equation}
где 
\begin{equation}\label{ed:m}
m(a, b) = \left\{ \begin{array}{ll}
    0, & \textrm{при $a = b$}\\
    1, & \textrm{иначе}
    \end{array} \right.
\end{equation}
Рассмотрим три алгоритма реализации нахождения расстояние Левенштейна.
\begin{enumerate}
    \item \textbf{Рекурсивный алгоритм}. Напрямую реализует формулу \ref{ed:D}.
    \item \textbf{Матричный (итеративный) алгоритм}. Представляет собой построчное заполнение матрицы [$N \times M$] промежуточными значениями $D(i, j)$.
    \item \textbf{Рекурсивный алгоритм с мемоизацией}. Оптимизирует рекурсивный алгоритм кэшированием уже вычисленных значений $D(i, j)$. В случае, если рекурсивный алгоритм выполняет прогон для данных, которые еще не были обработаны, результат нахождения расстояния заносится в кэш. В случае, если обработанные ранее данные встречаются снова, для них расстояние повторно не находится, а алгоритм переходит к следующему шагу.
\end{enumerate}

\section{Вычисление расстояния Дамерау--Левенштейна}

Расстояние Дамерау--Левенштейна вычисляется по приведенной ниже формуле, аналогичной формуле вычисления расстояния Левенштейна.

\begin{equation}
 D(i, j) =  \left\{
			\begin{aligned}
				&0, && i = 0, j = 0\\
		    	&i, && i > 0, j = 0\\
		    	&j, && i = 0, j > 0\\		    	
		    	&min \left\{
				\begin{aligned}
					&D(i, j - 1) + 1,\\
		            &D(i - 1, j) + 1,\\
		            &D(i - 1, j - 1) + m(S_{1}[i], S_{2}[i]), \\
		            &D(i - 2, j - 2) + m(S_{1}[i], S_{2}[i]),\\
		        \end{aligned} \right.
		        && 
				\begin{aligned}
					&, \text{ если } i, j > 0 \\
		            & \text{ и } S_{1}[i] = S_{2}[j - 1] \\
		            & \text{ и } S_{1}[i - 1] =  S_{2}[j] \\
		        \end{aligned} \\ 
		        &min \left\{
		        \begin{aligned}
		            &D(i, j - 1) + 1,\\
		            &D(i - 1, j) + 1, \\
		            &D(i - 1, j - 1) + m(S_{1}[i], S_{2}[i]),\\
		        \end{aligned} \right.  &&, \text{иначе}
			\end{aligned} \right.			
\end{equation}