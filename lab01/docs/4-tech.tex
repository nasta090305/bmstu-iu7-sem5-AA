\chapter{Технологический раздел}\label{ch:tech}

\section{Средства реализации}

Для реализации данной лабораторной был выбран язык Python, как простой язык с широким функционалом, содержащий все необходимые для работы инструменты и позволяющий завершить этап кодирования в кратчайшие сроки.
Для замеров времени используется функция process\_time() из модуля time(), возвращающая процессорное время.

\section{Программные модули}

Программа разбита на следующие модули:

\begin{itemize}
    \item \textbf{main.py} --- модуль, реализующий пользовательский интерфейс. Вызывает функции алгоритмов
    \item \textbf{funcs.py} --- модуль содержит функции поиска расстояния Левенштейна и Дамерау--Левенштейна. Функции итеративных алгоритмов с выводом матрицы работы вызываются в main.py, функции без вывода используются для замеров времени работы алгоритмов.
    \item \textbf{tests.py} --- модуль для проведения функционального тестирования.
    \item \textbf{time\_mes.py} --- модуль, реализующий замеры времени.
    \item \textbf{graph.py} --- модуль для графического отображения полученных замеров времени.
\end{itemize}

\section{Реализация алгоритмов}

В результате разработки алгоритмов были получены листинги \ref{lst:matrix_levenshtein.py} --- \ref{lsr:damerau-levenshtein.py}

\clearpage

\includelisting
    {matrix_levenshtein.py} % Имя файла с расширением (файл должен быть расположен в директории inc/lst/)
    {Исходный код итеративного алгоритма вычисления расстояния Левенштейна} % Подпись листинга

\clearpage

\includelisting
    {recurs_levenshtein.py}
    {Исходный код рекурсивного алгоритма вычисления расстояния Левенштейна}

\clearpage

\includelisting
    {memo_recurs_levenshtein.py}
    {Исходный код рекурсивного с мемоизацией алгоритма вычисления расстояния Левенштейна}

\clearpage

\includelisting
    {damerau_levenshtein.py}
    {Исходный код итеративного алгоритма вычисления расстояния Дамерау--Левенштейна}

\clearpage

\section{Функциональные тесты}

Функциональные тесты приведены в таблице \ref{table:tests} и были пройдены успешно.

\begin{table}[h]
\caption{Функциональные тесты}
\label{table:tests}
\begin{tabular}{|cc|cccc|}
\hline
\multicolumn{2}{|c|}{\textbf{Входные данные}} & \multicolumn{4}{c|}{\textbf{Полученное расстояние}} \\ \hline
\multicolumn{1}{|c|}{\multirow{2}{*}{\textbf{Строка 1}}} & \multirow{2}{*}{\textbf{Строка 2}} & \multicolumn{3}{c|}{\textbf{Левенштейн}} & \multirow{2}{*}{\textbf{\begin{tabular}[c]{@{}c@{}}Дамерау \\ Левенштейн\end{tabular}}} \\ \cline{3-5}
\multicolumn{1}{|c|}{} &  & \multicolumn{1}{c|}{\textbf{Итер.}} & \multicolumn{1}{c|}{\textbf{Рекурс.}} & \multicolumn{1}{c|}{\textbf{\begin{tabular}[c]{@{}c@{}}Рекурс.\\ с мемоиз.\end{tabular}}} &  \\ \hline
\multicolumn{1}{|c|}{а} & а & \multicolumn{1}{c|}{0} & \multicolumn{1}{c|}{0} & \multicolumn{1}{c|}{0} & 0 \\ \hline
\multicolumn{1}{|c|}{а} & б & \multicolumn{1}{c|}{1} & \multicolumn{1}{c|}{1} & \multicolumn{1}{c|}{1} & 1 \\ \hline
\multicolumn{1}{|c|}{вход} & вдох & \multicolumn{1}{c|}{2} & \multicolumn{1}{c|}{2} & \multicolumn{1}{c|}{2} & 2 \\ \hline
\multicolumn{1}{|c|}{кошка} & собака & \multicolumn{1}{c|}{3} & \multicolumn{1}{c|}{3} & \multicolumn{1}{c|}{3} & 3 \\ \hline
\multicolumn{1}{|c|}{абвгд} & авбдг & \multicolumn{1}{c|}{3} & \multicolumn{1}{c|}{3} & \multicolumn{1}{c|}{3} & 2 \\ \hline
\multicolumn{1}{|c|}{трон} & ртопа & \multicolumn{1}{c|}{4} & \multicolumn{1}{c|}{4} & \multicolumn{1}{c|}{4} & 3 \\ \hline
\multicolumn{1}{|c|}{пример} & примеры & \multicolumn{1}{c|}{1} & \multicolumn{1}{c|}{1} & \multicolumn{1}{c|}{1} & 1 \\ \hline
\end{tabular}
\end{table}