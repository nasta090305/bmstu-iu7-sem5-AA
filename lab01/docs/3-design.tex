\chapter{Конструкторский раздел}\label{ch:design}

\section{Требования к программе}

К программе предъявлены следующие требования:

\begin{itemize}
        \item вход: две строки
        \item выход: целое число --- расстояние Левенштейна/Дамерау--Левенштейна
	\item наличие пользовательского интерфейса для выбора действий;
	\item возможность вывода заполняемой матрицы для итеративных алгоритмов;
	\item наличие замера процессорного времени работы реализаций алгоритмов поиска расстояний Левенштейна и Дамерау--Левенштейна.
\end{itemize}

\section{Проектирование алгоритмов}

В процессе проектирования программы были разработаны схемы алгоритмов, представленные на рисунках \ref{img:matrix_levenshtein.drawio} --- \ref{img:damerau_levenshtein2.drawio}

\includeimage
    {matrix_levenshtein.drawio} % Имя файла без расширения (файл должен быть расположен в директории inc/img/)
    {f} % Обтекание (без обтекания)
    {h} % Положение рисунка (см. figure из пакета float)
    {0.54\textwidth} % Ширина рисунка
    {Схема итеративного алгоритма нахождения расстояния Левенштейна} % Подпись рисунка

\includeimage
    {recurs_levenshtein.drawio}
    {f}
    {h}
    {0.92\textwidth}
    {Схема рекурсивного алгоритма нахождения расстояния Левенштейна}

\includeimage
    {memo_levenshtein.drawio}
    {f}
    {h}
    {0.92\textwidth}
    {Схема рекурсивного c мемоизацией алгоритма нахождения расстояния Левенштейна}

\includeimage
    {memo_recurs_levenshtein.drawio}
    {f}
    {h}
    {1\textwidth}
    {Схема функции L из рекурсивного c мемоизацией алгоритма нахождения расстояния Левенштейна}

\includeimage
    {damerau_levenshtein.drawio}
    {f}
    {h}
    {0.6\textwidth}
    {Схема итеративного алгоритма нахождения расстояния Дамерау--Левенштейна. 1 часть}

\includeimage
    {damerau_levenshtein2.drawio}
    {f}
    {h}
    {0.7\textwidth}
    {Схема итеративного алгоритма нахождения расстояния Дамерау--Левенштейна. 2 часть}