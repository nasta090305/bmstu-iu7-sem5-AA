\chapter{Исследовательский раздел}\label{ch:research}

\section{Технические характеристики}

Технические характеристики устройства, на котором выполнялись замеры по времени, представлены далее.

\begin{itemize}
	\item Процессор: Intel(R) Core(TM) i5-8250U CPU @ 1.60GHz   1.80 GHz.
	\item Оперативная память: 8 ГБайт.
	\item Операционная система: Windows 11 Home версии 23H2.
\end{itemize}

При замерах времени ноутбук был включен в сеть электропитания и был нагружен только системными приложениями.

\section{Временные характеристики}

Замеры времени работы алгоритмов проводились на строках длиной от 1 до 100 символов для всех алгоритмов кроме рекурсивного алгоритма Левенштейна, так как он работает на порядки раз дольше остальных даже на размере строки до 10 символов, и дальнейшие замеры не имеют смысла и выполняются слишком долго. Для каждой длины строки замеры были выполнены 300 раз и получено среднее значение, занесенное в результат.

Результаты представлены в таблице \ref{table:time}. Прочерк в таблице означает, что замеры не проводились.

По приведенным результатам можно увидеть упомянутое выше значительное различие во времени работы рекурсивного алгоритма Левенштейна и остальных трёх. Рекурсивный с мемоизацией алгоритм Левенштейна в свою очередь в разы медленнее итеративного. Алгоритм Дамерау--Левенштейна же на небольших значениях отличается от итеративного незначительно, но с увеличением длины строки растет и разница во времени в силу наличия дополнительной проверки (при длине строки в 100 символов Дамерау--Левенштейн на 30\% медленнее обычного Левенштейна).

\clearpage

\begin{table}[h]
\caption{Результаты замеров времени работы алгоритмов}
\label{table:time}
\begin{tabular}{|c|cccc|}
\hline
\multirow{3}{*}{\begin{tabular}[c]{@{}c@{}}Длина строки,\\ символов\end{tabular}} & \multicolumn{4}{c|}{Время, мкс} \\ \cline{2-5} 
 & \multicolumn{3}{c|}{\textbf{Левенштейн}} & \multirow{2}{*}{\textbf{\begin{tabular}[c]{@{}c@{}}Дамерау-\\ Левенштейн\end{tabular}}} \\ \cline{2-4}
 & \multicolumn{1}{c|}{\textbf{Итер.}} & \multicolumn{1}{c|}{\textbf{Рекурс.}} & \multicolumn{1}{c|}{\textbf{\begin{tabular}[c]{@{}c@{}}Рекурс.\\ с мемоиз.\end{tabular}}} &  \\ \hline
1 & \multicolumn{1}{c|}{1.833} & \multicolumn{1}{c|}{1.316} & \multicolumn{1}{c|}{2.296} & 1.521 \\ \hline
2 & \multicolumn{1}{c|}{3.197} & \multicolumn{1}{c|}{5.313} & \multicolumn{1}{c|}{5.942} & 3.268 \\ \hline
3 & \multicolumn{1}{c|}{5.658} & \multicolumn{1}{c|}{27.062} & \multicolumn{1}{c|}{12.569} & 6.159 \\ \hline
4 & \multicolumn{1}{c|}{8.879} & \multicolumn{1}{c|}{142.767} & \multicolumn{1}{c|}{21.676} & 10.286 \\ \hline
5 & \multicolumn{1}{c|}{13.221} & \multicolumn{1}{c|}{770.604} & \multicolumn{1}{c|}{36.284} & 17.050 \\ \hline
6 & \multicolumn{1}{c|}{17.051} & \multicolumn{1}{c|}{3766.914} & \multicolumn{1}{c|}{46.917} & 21.460 \\ \hline
7 & \multicolumn{1}{c|}{21.546} & \multicolumn{1}{c|}{19985.579} & \multicolumn{1}{c|}{62.689} & 28.966 \\ \hline
8 & \multicolumn{1}{c|}{27.777} & \multicolumn{1}{c|}{108254.140} & \multicolumn{1}{c|}{87.913} & 38.761 \\ \hline
9 & \multicolumn{1}{c|}{35.359} & \multicolumn{1}{c|}{607252.302} & \multicolumn{1}{c|}{115.044} & 49.243 \\ \hline
10 & \multicolumn{1}{c|}{40.872} & \multicolumn{1}{c|}{3338151.763} & \multicolumn{1}{c|}{132.939} & 57.089 \\ \hline
20 & \multicolumn{1}{c|}{154.048} & \multicolumn{1}{c|}{-} & \multicolumn{1}{c|}{459.356} & 193.552 \\ \hline
30 & \multicolumn{1}{c|}{332.489} & \multicolumn{1}{c|}{-} & \multicolumn{1}{c|}{1024.083} & 424.310 \\ \hline
40 & \multicolumn{1}{c|}{578.928} & \multicolumn{1}{c|}{-} & \multicolumn{1}{c|}{1849.603} & 738.465 \\ \hline
50 & \multicolumn{1}{c|}{886.866} & \multicolumn{1}{c|}{-} & \multicolumn{1}{c|}{3173.269} & 1153.285 \\ \hline
60 & \multicolumn{1}{c|}{1304.971} & \multicolumn{1}{c|}{-} & \multicolumn{1}{c|}{4915.784} & 1674.950 \\ \hline
70 & \multicolumn{1}{c|}{1771.331} & \multicolumn{1}{c|}{-} & \multicolumn{1}{c|}{6825.604} & 2254.049 \\ \hline
80 & \multicolumn{1}{c|}{2292.528} & \multicolumn{1}{c|}{-} & \multicolumn{1}{c|}{8963.647} & 2938.029 \\ \hline
90 & \multicolumn{1}{c|}{2887.955} & \multicolumn{1}{c|}{-} & \multicolumn{1}{c|}{11290.257} & 3704.933 \\ \hline
100 & \multicolumn{1}{c|}{3595.309} & \multicolumn{1}{c|}{-} & \multicolumn{1}{c|}{14109.328} & 4667.696 \\ \hline
\end{tabular}
\end{table}

Те же выводы можно сделать, изучив графическое представление замеров времени, представленные на Рисунках \ref{img:all_time} и \ref{img:time2}

\begin{figure}[h]
    \centering
    \includegraphics[width=0.75\linewidth]{inc//img/all_time.png}
    \caption{Результаты замеры времени работы всех алгоритмов}
    \label{img:all_time}
\end{figure}

\begin{figure}[h]
    \centering
    \includegraphics[width=0.75\linewidth]{inc//img/time2.png}
    \caption{Результаты замеров без рекурсивного алгоритма на большем интервале}
    \label{img:time2}
\end{figure}

\clearpage

\section{Характеристики памяти}

В данной реализации итеративного алгоритма поиска расстояния Левенштейна в памяти хранятся 2 последних строки матрицы размером $(N + 1) \cdot (M + 1)$, где N - длина наименьшей входной строки, а М - наибольшей. В таком случае размер используемой памяти примерно равен $2 \cdot (N + 1) \cdot sizeof(int) $. 

Для итеративного алгоритма поиска расстояния Дамерау--Левенштейна ситуация аналогичная, только память выделяется на 3 последние строки в размере $3 \cdot (N + 1) \cdot sizeof(int) $. 

В рекурсивной реализации без мемоизации, для каждого рекурсивного вызова функции будет необходимо выделять память под стековый фрейм. Максимальная глубина стека вызовов на 2 больше суммы длин исходных строк. Следовательно, пиковая затрачиваемая память растет пропорционально сумме длин строк.

Рекурсивный алгоритм с мемоизацией выделяет память как под полную матрицу расстояний размером $(N + 1) \cdot (M + 1)$, так и под стековые фреймы для каждого вызова.