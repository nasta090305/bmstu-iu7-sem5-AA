\chapter*{ЗАКЛЮЧЕНИЕ}
\addcontentsline{toc}{chapter}{ЗАКЛЮЧЕНИЕ}

В результате исследования можно сделать вывод, что итеративные алгоритмы являются наиболее быстрыми из всех изученных, притом расстояние Левенштейна вычисляется быстрее расстояния Дамерау--Левенштейна. Память используемая итеративными алгоритмами пропорциональна длине наименьшей строки, а рекурсивным алгоритмом без мемоизации --- сумме двух входных строк. Рекурсивный алгоритм с мемоизацией же является самым затратным по памяти.

Таким образом, в результате выполнения лабораторной работы были изучены и реализованы алгоритмы поиска расстояния Левенштейна --- итерационный, рекурсивный и рекурсивный с мемоизацией, а также алгоритм поиска расстояния Дамерау--Левенштейна, и проведена оценка их ресурсной эффективности, то есть выполнены все поставленные задачи, и цель выполнения работы --- изучение алгоритмов поиска расстояний Левенштейна и Дамерау--Левенштейна --- можно считать достигнутой.