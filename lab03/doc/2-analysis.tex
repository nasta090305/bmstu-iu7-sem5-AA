\chapter{Аналитический раздел}\label{ch:analysis}

\section{Поиск полным перебором}

Алгоритм поиска элемента в массиве методом \textbf{полного перебора}, также известный как \textbf{линейный поиск}, используется для нахождения значения в массиве. Он проходит по каждому элементу, начиная с первого, и сравнивает его с искомым значением. Если элемент найден, возвращается его индекс; если нет — возвращается -1. 

Так как алгоритм перебирает элементы последовательно, доступ к элементам в начале массива будет получен быстрее, чем к элементам в конце. То есть лучшим будет случай, когда искомый элемент является первым элементом массива. В таком случае трудоёмкость алгоритма равна $O(1)$. Худшим же будет случай, когда искомый элемент последний или отсутствует в массиве. Трудоёмкость в таком случае будет равна $O(n)$. 

Этот метод подходит для небольших или неотсортированных массивов.

\section{Бинарный поиск}
\textbf{Бинарный поиск} ~\cite{binary} --- это эффективный алгоритм поиска элемента в отсортированном массиве. Он работает по принципу деления массива на две половины и сравнения искомого значения с элементом, находящимся в середине массива. Если искомое значение больше, то поиск продолжится в правой части массива, иначе --- в левой.

Так как начинаем сравнивать искомый элемент мы с середины массива, то лучшим случаем будет тот, где он находится в середине массива, и трудоёмкость будет равна $O(1)$. Худшим, с трудоёмкостью $O(log(n))$ соответственно, случай, где элемента нет в массиве, или он находится на позиции, для получения которой необходимо сделать $log(n)$ сравнений.

Бинарный поиск значительно быстрее линейного поиска для больших отсортированных массивов.