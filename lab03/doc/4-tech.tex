\chapter{Технологический раздел}\label{ch:tech}

\section{Требования к программе}

К программе предъявлены следующие требования:

\begin{itemize}
        \item на вход подается массив и элемент, который необходимо найти;
        \item результатом работы программы является целое число --- индекс искомого элемента в массиве, или -1 в случае его отсутствия;
	\item необходимо наличие пользовательского интерфейса для выбора действий и алгоритма работы;
	\item необходимо наличие замера количества сравнений, необходимых для поиска элемента в массиве для каждого из реализованных алгоритмов.
\end{itemize}

\section{Средства реализации}

Для реализации данной лабораторной был выбран язык Python ~\cite{python}, так как он содержит все необходимые для реализаций алгоритмов инструменты.

\section{Программные модули}

Программа разбита на следующие модули:

\begin{itemize}
    \item \textbf{main.py} --- модуль, реализующий пользовательский интерфейс и содержащий точку входа в программу;
    \item \textbf{funcs.py} --- модуль, содержащий функции алгоритмов поиска элемента в массиве;
    \item \textbf{cmp\_counter.py} --- модуль, содержащий позволяющие подсчитать количество сравнений функции алгоритмов поиска элемента в массиве

\end{itemize}

\section{Реализация алгоритмов}

Реализации алгоритмов представлены в листингах ~\ref{lst:linear.py} и ~\ref{lst:bin.py}.

\clearpage

\includelisting
{linear.py}
{Реализация алгоритма поиска элемента в массиве полным перебором}

\includelisting
{bin.py}
{Реализация алгоритма бинарного поиска элемента в массиве}

\clearpage

\section{Функциональные тесты}

Функциональные тесты приведены в таблице ~\ref{table:tests} и были пройдены успешно всеми реализациями алгоритмов.

\begin{table}[ht]
	\small
	\begin{center}
		\caption{Функциональные тесты}
		\label{table:tests}
		\begin{tabular}{|c|c|c|c|c|}
			\hline
			\bfseries Массив & \bfseries Элемент &\bfseries Индекс в массиве 
			& \multicolumn{2}{|c|}{\bfseries Количество сравнений} \\ \cline{4-5}
			& & & \bfseries Перебор & \bfseries Двоичный поиск \\ \hline
			1 2 3 4 5 6 & 1 & 0 & 1 & 3 \\ \hline
			1 2 3 4 5 6 & 0 & - & 6 & 3 \\ \hline
			1 2 3 4 5 6 & 4 & 3 & 4 & 1 \\ \hline
			1 2 3 4 5 6 & 6 & 5 & 6 & 3 \\ \hline
		\end{tabular}
	\end{center}
\end{table}
