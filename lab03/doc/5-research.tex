\chapter{Исследовательский раздел}\label{ch:research}

\section{Временные характеристики}

В качестве временной характеристики в этой работе используется количество сравнений, произведенное для поиска элемента. Для построения графиков были использованы функции библиотеки Matplotlib ~\cite{matplotlib}.

Исследование производится на массиве размером 1062.

На рисунках ~\ref{img:linear_cmp} и ~\ref{img:bin_cmp} представлены зависимости количества сравнений от позиции искомого элемента в массиве для поиска перебором и бинарного поиска соответственно. На рисунке ~\ref{img:sort_bin_cmp} представлена упорядоченная зависимость количества сравнений от позиции искомого элемента в массиве. 

\includeimage
{linear_cmp} % Имя файла без расширения (файл должен быть расположен в директории inc/img/)
{f} % Обтекание (без обтекания)
{H} % Положение рисунка (см. figure из пакета float)
{1\textwidth} % Ширина рисунка
{Гистограмма количества сравнений для поиска перебором} % Подпись рисунка

\includeimage
{bin_cmp} % Имя файла без расширения (файл должен быть расположен в директории inc/img/)
{f} % Обтекание (без обтекания)
{H} % Положение рисунка (см. figure из пакета float)
{1\textwidth} % Ширина рисунка
{Гистограмма количества сравнений для бинарного поиска} % Подпись рисунка

\includeimage
{sort_bin_cmp} % Имя файла без расширения (файл должен быть расположен в директории inc/img/)
{f} % Обтекание (без обтекания)
{H} % Положение рисунка (см. figure из пакета float)
{1\textwidth} % Ширина рисунка
{Упорядоченная гистограмма количества сравнений для бинарного поиска} % Подпись рисунка

\clearpage

