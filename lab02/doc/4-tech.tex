\chapter{Технологический раздел}\label{ch:tech}

\section{Требования к программе}

К программе предъявлены следующие требования:

\begin{itemize}
        \item на вход подаются две матрицы и их размеры;
        \item результатом работы программы является матрица;
	\item необходимо наличие пользовательского интерфейса для выбора действий и алгоритма работы;
	\item необходимо наличие замера процессорного времени работы реализованных алгоритмов умножения матриц.
\end{itemize}

\section{Средства реализации}

Для реализации данной лабораторной был выбран язык Python ~\cite{python}, так как он содержит все необходимые для реализаций алгоритмов инструменты и имеет модуль utime() ~\cite{python_utime} для замеров времени.
Для замеров времени используется функция ticks\_us() из модуля utime().

\section{Программные модули}

Программа разбита на следующие модули:

\begin{itemize}
    \item \textbf{main.py} --- модуль, реализующий пользовательский интерфейс;
    \item \textbf{funcs.py} --- модуль содержит функции умножения, ввода и вывода матриц;
    \item \textbf{tests.py} --- модуль для проведения функционального тестирования;
    \item \textbf{time\_mes.py} --- модуль, реализующий замеры времени;
    \item \textbf{graph.py} --- модуль для графического отображения полученных замеров времени.
\end{itemize}

\section{Реализация алгоритмов}

Реализации алгоритмов представлены в листингах ~\ref{lst:standart.py} --- ~\ref{lst:opt_winograd.py}

\clearpage

\includelisting
{standart.py}
{Классический алгоритм умножения матриц}

\clearpage

\includelisting
{winograd.py}
{Алгоритм Винограда}

\clearpage

\includelisting
{opt_winograd.py}
{Оптимизированный алгоритм Винограда}

\clearpage

\section{Функциональные тесты}

Функциональные тесты приведены в таблице ~\ref{table:tests} и были пройдены успешно всеми алгоритмами.

\begin{table}[h]
\centering
\caption{Функциональные тесты}
\label{table:tests}
\begin{tabular}{|cc|c|}
\hline
\multicolumn{2}{|c|}{\textbf{Входные данные}} & \multirow{2}{*}{\textbf{\begin{tabular}[c]{@{}c@{}}Ожидаемый результат\end{tabular}}} \\ \cline{1-2}
\multicolumn{1}{|c|}{\textbf{Матрица А}} & \textbf{Матрица В} &  \\ \hline
\multicolumn{1}{|c|}{$\begin{pmatrix} 2 \end{pmatrix}$} & $\begin{pmatrix} 3 \end{pmatrix}$ & $\begin{pmatrix} 6 \end{pmatrix}$ \\ \hline
\multicolumn{1}{|c|}{\begin{tabular}[c]{@{}c@{}}$\begin{pmatrix} 1 & 2\\ 3 & 4 \end{pmatrix}$\end{tabular}} & \begin{tabular}[c]{@{}c@{}}$\begin{pmatrix} 5 & 6\\ 7 & 8 \end{pmatrix}$\end{tabular} & \begin{tabular}[c]{@{}c@{}}$\begin{pmatrix} 19 & 22\\ 43 & 50 \end{pmatrix}$\end{tabular} \\ \hline
\multicolumn{1}{|c|}{$\begin{pmatrix} 1 2 3 \end{pmatrix}$} & \begin{tabular}[c]{@{}c@{}}$\begin{pmatrix} 0 & 6\\ -9 & 5\\ -8 & 7 \end{pmatrix}$\end{tabular} & $\begin{pmatrix} -42 & 37 \end{pmatrix}$ \\ \hline
\multicolumn{1}{|c|}{$\begin{pmatrix} 1 2 3 \end{pmatrix}$} & \begin{tabular}[c]{@{}c@{}}$\begin{pmatrix} 0 & 6\\ -9 & 5\end{pmatrix}$\end{tabular} & None \\ \hline
\multicolumn{1}{|c|}{$\begin{pmatrix} 1 2 3 \end{pmatrix}$} & $\begin{pmatrix} & \end{pmatrix}$ & None \\ \hline
\end{tabular}
\end{table}