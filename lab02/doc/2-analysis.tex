\chapter{Аналитический раздел}\label{ch:analysis}

\section{Определение матрицы}

\textbf{Матрицей} ~\cite{matrix} называют таблицу чисел $a_{ik}$ вида
\begin{equation}
	\begin{pmatrix}
		a_{11} & a_{12} & \ldots & a_{1n}\\
		a_{21} & a_{22} & \ldots & a_{2n}\\
		\vdots & \vdots & \ddots & \vdots\\
		a_{m1} & a_{m2} & \ldots & a_{mn}
	\end{pmatrix},
\end{equation}

состоящую из $m$ строк и $n$ столбцов.

Пусть $A$ -- матрица, тогда $A_{i,j}$ -- элемент этой матрицы, который находится на \textit{i-ой} строке и \textit{j-ом} столбце.

\section{Классический алгоритм умножения матриц}
Пусть даны две матрицы

\begin{equation}
	A_{lm} = \begin{pmatrix}
		a_{11} & a_{12} & \ldots & a_{1m}\\
		a_{21} & a_{22} & \ldots & a_{2m}\\
		\vdots & \vdots & \ddots & \vdots\\
		a_{l1} & a_{l2} & \ldots & a_{lm}
	\end{pmatrix},
	\quad
	B_{mn} = \begin{pmatrix}
		b_{11} & b_{12} & \ldots & b_{1n}\\
		b_{21} & b_{22} & \ldots & b_{2n}\\
		\vdots & \vdots & \ddots & \vdots\\
		b_{m1} & b_{m2} & \ldots & b_{mn}
	\end{pmatrix},
\end{equation}

тогда матрица $C$
\begin{equation}
	C_{ln} = \begin{pmatrix}
		c_{11} & c_{12} & \ldots & c_{1n}\\
		c_{21} & c_{22} & \ldots & c_{2n}\\
		\vdots & \vdots & \ddots & \vdots\\
		c_{l1} & c_{l2} & \ldots & c_{ln}
	\end{pmatrix},
\end{equation}

где
\begin{equation}
	\label{eq:M}
	c_{ij} =
	\sum_{r=1}^{m} a_{ir}b_{rj} \quad (i=\overline{1,l}; j=\overline{1,n})
\end{equation}

будет называться произведением матриц $A$ и $B$.

Классический алгоритм умножения матриц реализует эту формулу.

\section{Алгоритм Винограда}

Алгоритм Винограда ~\cite{winograd} - алгоритм умножения матриц, основанный на приведенных ниже рассуждениях.

Пусть есть два вектора $V = (v1, v2, v3, v4)$ и $W = (w1, w2, w3, w4)$.  

Их скалярное произведение равно ~(\ref{form:vec_mul}) 

\begin{equation} \label{form:vec_mul}
V \cdot W=v_1 \cdot w_1 + v_2 \cdot w_2 + v_3 \cdot w_3 + v_4 \cdot w_4
\end{equation}

Равенство ~(\ref{form:vec_mul}) можно переписать в виде ~(\ref{form:vino_mul}) 
\begin{equation} \label{form:vino_mul}
V \cdot W=(v_1 + w_2) \cdot (v_2 + w_1) + (v_3 + w_4) \cdot (v_4 + w_3) - v_1 \cdot v_2 - v_3 \cdot v_4 - w_1 \cdot w_2 - w_3 \cdot w_4
\end{equation}

Выражение в правой части формулы ~(\ref{form:vino_mul}) допускает предварительную обработку: его части можно вычислить заранее и запомнить для каждой строки первой матрицы и для каждого столбца второй, что позволяет выполнять для каждого элемента лишь первые два умножения и последующие пять сложений, а также дополнительно два сложения. При нечетном значении размера матрицы нужно дополнительно добавить произведения крайних элементов соответствующих строк и столбцов к результату. 

\section{Оптимизации алгоритма Винограда}

В рассмотренный выше алгоритм Винограда при программной реализации можно внести следующие оптимизации:

\begin{enumerate}
    \item инкремент счётчика наиболее вложенного цикла на 2; 
    \item объединение III и IV частей алгоритма Винограда; 
    \item введение декремента при вычислении вспомогательных массивов.
\end{enumerate}